\documentclass[a4paper,10pt]{article}
%\documentclass[a4paper,10pt]{scrartcl}

\usepackage[MeX]{polski}
\usepackage[utf8]{inputenc}

\usepackage{hyperref}
\hypersetup{
    colorlinks,
    citecolor=black,
    filecolor=black,
    linkcolor=black,
    urlcolor=black
}

\title{Metody Realizacji Języków Programowania 2011 \newline Projekt semestralny - Sprawozdanie}
\author{Błażej Owczarczyk}
\date{\today}

\setcounter{tocdepth}{3}

\pdfinfo{%
  /Title    (Sprawozdanie)
  /Author   (Błażej Owczarczyk)
  /Creator  ()
  /Producer ()
  /Subject  ()
  /Keywords ()
}

\begin{document}
\maketitle
 \begin{center} 
  Pod kierunkiem prof. Andrzeja Salwickiego
 \end{center}
\newpage

\tableofcontents
\newpage

\section{Co było celem?}
Celem projektu było:
\begin{enumerate}
 \item Zdefiniowanie języka programowania zawierającego:
 \begin{enumerate}
  \item zmienne
  \item instrukcje warunkowe
  \item pętle.
 \end{enumerate}
 \item Napisanie gramatyki.
 \item Stworzenie generatora kodu pośredniego używając dostępnych narzędzi.
 \item Stworzenie maszyny wirtualnej interpretującej kod pośredni.
\end{enumerate}

\newpage

\section{Język Blang}

\subsection{Opis}
Stworzono język Blang, w którym:
\begin{enumerate}
 \item Zmienne nie muszą być deklarowane.
 \item Domyślny i jedyny typ zmiennych to liczby całkowite (integer).
 \item Wyrażenia są uznawane za prawdziwe, jeśli ich wartość jest różna od zera.
 \item Dostępne operatory to:
 \begin{enumerate}
  \item '+' - dodawanie
  \item '-' - odejmowanie
  \item '*' - mnożenie
  \item '\textbackslash' - dzielenie
  \item 'mod' - dzielenie modulo
 \end{enumerate}
\end{enumerate}

\subsection{Składnia}
Składnia języka Blang:
\begin{enumerate}
 \item Program
  \begin{verbatim}
   program
     {instrukcje}
     return {wyrażenie};
   endprogram
  \end{verbatim}
 \item Instrukcje
 \begin{enumerate}
  \item instrukcja przypisania
   \begin{verbatim}
    {identyfikator} := {wyrażenie};
   \end{verbatim}
  \item instrukcja warunkowa
   \begin{verbatim}
    if {wyrażenie} then
      {instrukcje}
    else
      {instrukcje}
    end if;
   \end{verbatim}
  \item instrukcja pętli
   \begin{verbatim}
    while {wyrażenie} do
      {instrukcje}
    end while;
   \end{verbatim}
 \end{enumerate}
\end{enumerate}

\newpage

\section{Przykłady}

\subsection{przykładowy program}
 \begin{verbatim}
  program
    n := 51;

    while n - 1 do
      if n mod 2 then
        n := n * 3 + 1;
      else
        n := n \ 2;
      end if;
    end while;

    return n;
  endprogram
 \end{verbatim}

\subsection{program obliczający potęgi}
 \begin{verbatim}
  program
    base := 2;
    exponent := 12;
    result := 1;

    while exponent do
      result := result * base;
      exponent := exponent - 1;
    end while;

    return result;
  endprogram
 \end{verbatim}

\newpage

\section{Podręcznik użytkownika}
Aby uruchomić program w języku Blang należy:
\begin{enumerate}
 \item Napisać program w edytorze tekstu (np. notepad w systemie windows lub vi w linuxie)
 \item Zapisać program w pliku tekstowym (jako plain text) z rozszerzeniem .blang
 \item Uruchomić konsolę i wpisać
  \begin{verbatim}
   java -jar {path/to/blang}/Blang.jar {nazwa-pliku}.blang
  \end{verbatim}
  gdzie $\{$path/to/blang$\}$ to ścieżka do katalogu, w którym znajduje się plik Blang.jar, a $\{$nazwa-pliku$\}$ to nazwa stworzonego pliku z programem.
 \item Na ekranie pojawi się wiadomość powitalna, treść programu oraz wartość wyrażenia podanego po słowie kluczowym return.
\end{enumerate}

\end{document}
